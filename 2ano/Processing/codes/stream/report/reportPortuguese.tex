\documentclass[
	12pt,				% tamanho da fonte
	oneside,			% para impressão em recto e verso. Oposto a oneside
	a4paper,			% tamanho do papel. 
	english,			% idioma adicional para hifenização
	brazil,				% o último idioma é o principal do documento
	]{abntex2}

% ---
% Pacotes fundamentais 
% ---
\usepackage{lmodern}			% Usa a fonte Latin Modern
\usepackage[T1]{fontenc}		% Selecao de codigos de fonte.
\usepackage[utf8]{inputenc}		% Codificacao do documento (conversão automática dos acentos)
\usepackage{indentfirst}		% Indenta o primeiro parágrafo de cada seção.
\usepackage{color}				% Controle das cores
\usepackage{graphicx}			% Inclusão de gráficos
\usepackage{microtype} 			% para melhorias de justificação
\usepackage{multicol}
\usepackage{multirow}
\usepackage[brazilian,hyperpageref]{backref}	 % Paginas com as citações na bibl
\usepackage[alf]{abntex2cite}	% Citações padrão ABNT
\usepackage{float}
% --- 
% CONFIGURAÇÕES DE PACOTES
% --- 

% ---
% Configurações do pacote backref
% Usado sem a opção hyperpageref de backref
\renewcommand{\backrefpagesname}{Citado na(s) página(s):~}
% Texto padrão antes do número das páginas
\renewcommand{\backref}{}
% Define os textos da citação
\renewcommand*{\backrefalt}[4]{
	\ifcase #1 %
		Nenhuma citação no texto.%
	\or
		Citado na página #2.%
	\else
		Citado #1 vezes nas páginas #2.%
	\fi}%
% ---

% ---
% Informações de dados para CAPA e FOLHA DE ROSTO
% ---
    \titulo{Prática 22: Stream em Java}
\autor{Pedro Inácio Rodrigues Pontes}
\local{Belo Horizonte, Brasil}
\data{2024}
\instituicao{%
  Universidade Federal de Minas Gerais
  \par
  Colégio Técnico
  \par
  Curso Técnico em Desenvolvimento de Sistemas}

\definecolor{blue}{RGB}{41,5,195}

\makeatletter
\hypersetup{
     	%pagebackref=true,
		pdftitle={\@title}, 
		pdfauthor={\@author},
    	pdfsubject={\imprimirpreambulo},
		colorlinks=true,       		% false: boxed links; true: colored links
    	linkcolor=blue,          	% color of internal links
    	citecolor=blue,        		% color of links to bibliography
    	filecolor=magenta,      		% color of file links
		urlcolor=blue,
		bookmarksdepth=4
}
\makeatother

\renewcommand{\thesection}{\arabic{section}}
\setlength{\parindent}{1.3cm}
\setlength{\parskip}{0.2cm} 

\makeindex


\begin{document}

\selectlanguage{brazil}
\frenchspacing 

\imprimircapa

{
\ABNTEXchapterfont

\textual

% ----------------------------------------------------------
% Introdução (exemplo de capítulo sem numeração, mas presente no Sumário)
% ----------------------------------------------------------
\section{Introdução}

O objetivo da atividade é, por meio do uso de streams, remover as partículas do sistema de partículas dado que possuem a vida menor ou igual a 0.

\section{Desenvolvimento}

Foi adicionado o seguinte código para tal fim:

\begin{verbatim}
List<Particula> particulasParaRemover = particulas.stream()
                                          .filter(p -> p.duracaoVida <= 0)
                                          .collect(Collectors.toList());
particulas.removeAll(particulasParaRemover);
\end{verbatim}

É criada uma lista de partículas para remover, a qual vem de uma operação stream onde é usado o filter para filtrar apenas as partículas que tem vida menor ou igual a zero. Depois, a partir do método removeAll, são removidas as ocorrências das partículas que estavam na lista de partículas para remover no sistema de partículas.

\section{Resultados}

\begin{figure}[H]
\centering
\includegraphics[width=0.9\textwidth]{imgs/img1.png}
\caption{}
\label{imagem 1}
\end{figure}
\begin{figure}[H]
\centering
\includegraphics[width=0.9\textwidth]{imgs/img2.png}
\caption{}
\label{imagem 2}
\end{figure}

Perceba que nessas duas imagens, as partículas continuam sendo exibidas, isso ocorre porque as mortas estão sendo retiradas, mostrando o sucesso do código.
\section{Conclusão}

Os resultados ocorreram de acordo com o esperado. As partículas com vida menor ou igual a zero foram retiradas utilizando streams e a lógica desse uso foi entendida.

\end{document}

