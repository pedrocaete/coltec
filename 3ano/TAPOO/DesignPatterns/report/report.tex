\documentclass[
	12pt,				% tamanho da fonte
	oneside,			% para impressão em recto e verso. Oposto a oneside
	a4paper,			% tamanho do papel. 
	english,			% idioma adicional para hifenização
	brazil,				% o último idioma é o principal do documento
	]{abntex2}

% ---
% Pacotes fundamentais 
% ---
\usepackage{lmodern}			% Usa a fonte Latin Modern
\usepackage[T1]{fontenc}		% Selecao de codigos de fonte.
\usepackage[utf8]{inputenc}		% Codificacao do documento (conversão automática dos acentos)
\usepackage{indentfirst}		% Indenta o primeiro parágrafo de cada seção.
\usepackage{color}				% Controle das cores
\usepackage{graphicx}			% Inclusão de gráficos
\usepackage{microtype} 			% para melhorias de justificação
\usepackage{multicol}
\usepackage{multirow}
\usepackage[brazilian,hyperpageref]{backref}	 % Paginas com as citações na bibl
\usepackage[alf]{abntex2cite}	% Citações padrão ABNT
\usepackage{listings}
\usepackage{float}

\lstset{
language=C,
basicstyle=\ttfamily\footnotesize,
keywordstyle=\color{blue},
commentstyle=\color{green},
stringstyle=\color{red},
numbers=left,
numberstyle=\tiny,
stepnumber=1,
numbersep=5pt,
backgroundcolor=\color{lightgray!10},
showspaces=false,
showstringspaces=false,
showtabs=false,
frame=single,
rulecolor=\color{black},
tabsize=2,
captionpos=b,
breaklines=true,
breakatwhitespace=false,
escapeinside={%}{)}
}

% --- 
% CONFIGURAÇÕES DE PACOTES
% --- 

% ---
% Configurações do pacote backref
% Usado sem a opção hyperpageref de backref
\renewcommand{\backrefpagesname}{Citado na(s) página(s):~}
% Texto padrão antes do número das páginas
\renewcommand{\backref}{}
% Define os textos da citação
\renewcommand*{\backrefalt}[4]{
	\ifcase #1 %
		Nenhuma citação no texto.%
	\or
		Citado na página #2.%
	\else
		Citado #1 vezes nas páginas #2.%
	\fi}%
% ---

% ---
% Informações de dados para CAPA e FOLHA DE ROSTO
% ---
\titulo{Design Patterns: Sistema Eccomerce}
\autor{Pedro Inácio Rodrigues Pontes}
\local{Belo Horizonte, Brasil}
\data{2025}
\instituicao{%
  Universidade Federal de Minas Gerais
  \par
  Colégio Técnico
  \par
  Curso Técnico em Desenvolvimento de Sistemas}

\definecolor{blue}{RGB}{41,5,195}

\makeatletter
\hypersetup{
     	%pagebackref=true,
		pdftitle={\@title}, 
		pdfauthor={\@author},
    	pdfsubject={\imprimirpreambulo},
		colorlinks=true,       		% false: boxed links; true: colored links
    	linkcolor=blue,          	% color of internal links
    	citecolor=blue,        		% color of links to bibliography
    	filecolor=magenta,      		% color of file links
		urlcolor=blue,
		bookmarksdepth=4
}
\makeatother

\renewcommand{\thesection}{\arabic{section}}
\setlength{\parindent}{1.3cm}
\setlength{\parskip}{0.2cm} 

\makeindex


\begin{document}

\selectlanguage{brazil}
\frenchspacing 

\imprimircapa

{
\ABNTEXchapterfont

\textual

% ----------------------------------------------------------
% Introdução (exemplo de capítulo sem numeração, mas presente no Sumário)
% ----------------------------------------------------------
\section{Introdução}
O objetivo do presente trabalho foi a implementação de um sistema de e-commerce simplificado em C# que demonstrasse a aplicação prática de cinco padrões de projeto fundamentais da engenharia de software. O sistema original precisava gerenciar múltiplas funcionalidades essenciais para o comércio eletrônico - incluindo criação de produtos, processamento de pagamentos, notificações e configurações globais -, onde o desafio principal estava em criar uma arquitetura flexível e de fácil manutenção.
O desafio consistia em transformar requisitos funcionais complexos em uma implementação modular utilizando padrões de projeto (\textit{Design Patterns}), com o objetivo de criar componentes independentes e reutilizáveis. Para isso, foi necessário implementar cinco padrões específicos: \textit{Singleton} para gerenciamento de configurações, \textit{Factory Method} para criação de produtos, \textit{Observer} para sistema de notificações, \textit{Strategy} para métodos de pagamento e \textit{Decorator} para funcionalidades extras dos produtos.
A principal motivação para esta abordagem era resolver problemas comuns em sistemas de e-commerce, como rigidez arquitetural, dificuldade de manutenção e baixa capacidade de extensão, criando um sistema onde novas funcionalidades pudessem ser adicionadas sem modificar o código existente.
\section{Desenvolvimento}
\subsection{Arquitetura Base do Sistema}
O sistema de e-commerce foi desenvolvido seguindo uma arquitetura modular onde cada padrão de projeto resolve um problema específico. A implementação base conta com classes abstratas e interfaces que definem contratos claros para extensibilidade futura, permitindo que novas funcionalidades sejam adicionadas sem modificar o código existente.
\subsubsection{Estrutura Fundamental dos Produtos}
A hierarquia de produtos foi implementada através de uma classe abstrata base que define o comportamento comum a todos os produtos:
\begin{lstlisting}[language=C]
public abstract class Produto
{
public string Nome { get; set; }
public decimal Preco { get; set; }
public abstract string ObterCategoria();
public abstract decimal CalcularFrete();
}
\end{lstlisting}
Cada tipo de produto concreto implementa suas próprias regras de negócio. A classe \textit{Eletronico} define sua categoria como "Eletrônicos" e calcula o frete como 5% do preço do produto, enquanto a classe \textit{Roupa} possui uma propriedade específica para tamanho e frete fixo de R$12,50. A classe \textit{Livro} inclui propriedades para autor e número de páginas, com frete condicional baseado no tamanho da obra.
\subsection{Implementação dos Padrões de Projeto}
\subsubsection{Padrão Singleton - Gerenciamento de Configurações}
O padrão Singleton foi implementado para garantir uma única instância do gerenciador de configurações em todo o sistema. A implementação utiliza \textit{double-checked locking} para thread-safety:
\begin{lstlisting}[language=C]
public sealed class GerenciadorConfiguracao
{
private static GerenciadorConfiguracao _instancia;
private static readonly object _bloqueio = new object();
private GerenciadorConfiguracao() { }

public static GerenciadorConfiguracao Instancia
{
    get
    {
        if (_instancia == null)
        {
            lock (_bloqueio)
            {
                if (_instancia == null)
                    _instancia = new GerenciadorConfiguracao();
            }
        }
        return _instancia;
    }
}

public string ConexaoBancoDados { get; set; } = "ConexaoPadrao";
public decimal TaxaImposto { get; set; } = 0.08m;
}
\end{lstlisting}
\subsubsection{Padrão Factory Method - Criação de Produtos}
O Factory Method foi implementado para encapsular a lógica de criação de diferentes tipos de produtos. Cada fábrica concreta é responsável por criar produtos específicos com suas características particulares:
\begin{lstlisting}[language=C]
public class FabricaEletronicos : FabricaProduto
{
public override Produto CriarProduto(string nome, decimal preco)
{
return new Eletronico { Nome = nome, Preco = preco };
}
}
public class FabricaLivro : FabricaProduto
{
public override Produto CriarProduto(string nome, decimal preco)
{
return CriarProduto(nome, preco, "Autor Desconhecido", 100);
}
public Produto CriarProduto(string nome, decimal preco, string autor, int numeroPaginas)
{
    return new Livro
    {
        Nome = nome,
        Preco = preco,
        Autor = autor,
        NumeroPaginas = numeroPaginas,
    };
}
}
\end{lstlisting}
\subsubsection{Padrão Observer - Sistema de Notificações}
O padrão Observer foi implementado para criar um sistema de notificações reativo que informa automaticamente sobre mudanças no status dos pedidos:
\begin{lstlisting}[language=C]
public class Pedido : IObservavel<IObservadorPedido>
{
private List<IObservadorPedido> _observadores = new List<IObservadorPedido>();
private string _status;
public string Status
{
    get => _status;
    set
    {
        _status = value;
        NotificarObservadores();
    }
}

public void Inscrever(IObservadorPedido observador)
{
    _observadores.Add(observador);
}

private void NotificarObservadores()
{
    foreach (var observador in _observadores)
    {
        observador.AoMudarStatusPedido(this, _status);
    }
}
}
\end{lstlisting}
\subsubsection{Padrão Strategy - Métodos de Pagamento}
O padrão Strategy foi implementado para suportar múltiplas formas de pagamento com validações específicas para cada método. Cada estratégia implementa sua própria lógica de processamento:
\begin{lstlisting}[language=C]
public class PagamentoCartaoCredito : IEstrategiaPagamento
{
public string NumeroCartao { get; set; }
public string NomeTitular { get; set; }
public bool ProcessarPagamento(decimal valor)
{
    return valor > 0 && valor < 5000;
}

public string ObterDetalhesPagamento()
{
    string Ultimos4Digitos = NumeroCartao.Substring(NumeroCartao.Length - 4);
    return $"Últimos 4 Dígitos Cartão de Crédito: {Ultimos4Digitos}";
}
}
\end{lstlisting}
\subsubsection{Padrão Decorator - Funcionalidades Extras}
O padrão Decorator foi implementado para permitir a adição dinâmica de funcionalidades extras aos produtos sem modificar suas classes base. Cada decorador adiciona comportamentos específicos:
\begin{lstlisting}[language=C]
public class DecoradorGarantia : DecoradorProduto
{
private int _mesesGarantia;
public DecoradorGarantia(Produto produto, int mesesGarantia) : base(produto)
{
    _mesesGarantia = mesesGarantia;
    Preco = produto.Preco + (mesesGarantia * 10); // R$10 por mês
}

public override string ObterCategoria() => base.ObterCategoria() + $" Garantia de {_mesesGarantia}";
}
\end{lstlisting}
\subsection{Integração e Fluxo Principal}
O sistema integra todos os padrões através da classe principal \textit{SistemaECommerce}, que demonstra o fluxo completo desde a configuração inicial até o processamento final do pagamento. A implementação garante que cada componente funcione independentemente, mas se integre harmoniosamente com os demais, proporcionando uma experiência de uso fluida e extensível.
\section{Resultados}

\begin{verbatim}
Configuração: Conexão=ServidorLocal;Banco=EcommerceDB, Imposto=10.00 %
[EMAIL] O pedido mudou para o status: Processando
[SMS] O pedido mudou para o status: Processando
[EMAIL] O pedido mudou para o status: Enviado
[SMS] O pedido mudou para o status: Enviado
[EMAIL] O pedido mudou para o status: Entregue
[SMS] O pedido mudou para o status: Entregue
1134.99
True
Pagamento Cartão: Aprovado - Últimos 4 Dígitos Cartão de Crédito: 5678
Pagamento PayPal: Aprovado - Pagamento com PayPal 
 Email: usuario@paypal.com
Pagamento Pix: Aprovado - PIX: 99999999999
\end{verbatim}

\section{Conclusão}
A implementação do sistema de e-commerce com padrões de projeto foi realizada com sucesso. O sistema demonstrou como a aplicação adequada de design patterns pode resolver problemas arquiteturais complexos, proporcionando uma base sólida para futuras extensões e manutenções. Cada um dos cinco padrões implementados atendeu aos requisitos específicos propostos, criando uma arquitetura modular e flexível.
Os resultados demonstraram que os padrões de projeto são mais eficazes quando aplicados em conjunto, formando uma arquitetura coesa. O padrão Singleton garantiu o gerenciamento centralizado de configurações, o Factory Method proporcionou flexibilidade na criação de produtos, o Observer implementou um sistema de notificações reativo, o Strategy permitiu múltiplos métodos de pagamento, e o Decorator adicionou funcionalidades extras de forma dinâmica.
A integração entre os padrões mostrou-se harmoniosa, com cada componente funcionando independentemente mas contribuindo para o sistema como um todo. A implementação do fluxo principal na classe \textit{SistemaECommerce} demonstrou como produtos podem ser criados via Factory, decorados com funcionalidades extras, processados através de diferentes estratégias de pagamento, e ter suas mudanças de status notificadas automaticamente aos observadores interessados.
A solução implementada para a separação de responsabilidades através dos padrões foi fundamental para criar um sistema extensível e de fácil manutenção. Esta abordagem permitiu que cada funcionalidade fosse implementada de forma independente, facilitando testes unitários e futuras modificações, o que é essencial para a evolução contínua de sistemas comerciais.
A arquitetura baseada em padrões de projeto proporcionou benefícios substanciais em termos de organização do código, reutilização de componentes e facilidade de extensão. O sistema se mostrou preparado para crescer e adaptar-se a novos requisitos, onde novas formas de pagamento, tipos de produtos ou métodos de notificação podem ser adicionados sem impactar o código existente. Os resultados confirmam que, embora os padrões de projeto introduzam complexidade inicial, eles são viáveis e vantajosos para aplicações comerciais que precisam de flexibilidade e manutenibilidade a longo prazo.

\end{document}
